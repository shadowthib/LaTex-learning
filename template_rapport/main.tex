\documentclass[a4paper, 12pt]{article}
\usepackage[utf8]{inputenc}
\usepackage[T1]{fontenc}
\usepackage[french]{babel}
\usepackage{graphicx}
\usepackage{amsmath}
\usepackage{hyperref}
\usepackage{geometry}
\usepackage{comment}
\usepackage{listings}
\usepackage[none]{hyphenat}
\usepackage{fancyhdr}
\usepackage{datetime}
%\usepackage{draftwatermark}

\geometry{left=2cm, right=2cm, top=2cm, bottom=2cm}

% Configuration du watermark si besoin décommenter les lignes suivantes
%\SetWatermarkText{Confidential}
%\SetWatermarkScale{1.5}
%\SetWatermarkColor{red}
%\SetWatermarkAngle{45}

\fancypagestyle{couverture}{
    \fancyhf{}
    \fancyhead[L]{\includegraphics[width=8cm]{img/logo_hehbe_tech.png}} 
    \fancyhead[R]{\vspace{1.8cm}\textbf{Année académique 20xx-20xx}} 
    \fancyfoot[L]{Promoteur : Nom Prénom} 
    \fancyfoot[R]{Étudiant : Nom Prénom} 
    \fancyfoot[C]{} 
    \renewcommand{\headrulewidth}{0pt} 
    \renewcommand{\footrulewidth}{0pt} 
}

\fancypagestyle{page_de_garde}{
    \fancyhf{}
    \fancyhead[L]{\includegraphics[width=8cm]{img/logo_hehbe_tech.png}} 
    \fancyhead[R]{\vspace{1.8cm}\textbf{Année académique 20xx-20xx}} 
    \fancyfoot[L]{Promoteur : Nom Prénom} 
    \fancyfoot[C]{\monthname~\the\year}
    \fancyfoot[R]{Étudiant : Nom Prénom}
    \renewcommand{\headrulewidth}{0pt} 
    \renewcommand{\footrulewidth}{0pt} 
}

\pagestyle{fancy}
\fancyhf{}
\renewcommand{\footrulewidth}{0.1pt}
\renewcommand{\headrulewidth}{0.1pt}
\fancyhead[L]{\includegraphics[width=0.15\textwidth]{img/logo_hehbe_tech.png}}
\fancyhead[R]{\vspace{0.1cm}\textbf{Année académique 2024-2025}}
\fancyfoot[L]{ \today}
\fancyfoot[C]{Page \thepage}
\fancyfoot[R]{Version N° 0.5}
\headsep = 2cm
\textheight = 24cm



\newcommand{\premierepage}{
     \thispagestyle{couverture}
        \vspace*{2cm}
        \begin{center}    
            {\large \textbf{HAUTE ECOLE DE LA COMMUNAUTE FRANCAISE EN HAINAUT}}\\
            {\large Département Sciences et Technologies}\\
            {\large 8A Avenue Victor Maistriau -- 7000 Mons}
        \end{center}
        \vspace{3cm}
        \begin{center}
            \fboxrule=0.3mm
            \fbox{
                \parbox{\textwidth}{
                    \vspace{1cm}
                    \centering
                    \textbf{Lorem ipsum dolor sit amet, consectetur adipiscing elit. Vivamus lacinia odio vitae}
                    \vspace{1cm}
                }
            }
        \end{center}
        \vspace{1.5cm}
        \begin{center}
            \normalsize{\textit{Travail de fin d'études réalisé en vue de l'obtention du titre de bachelier en Informatique et systèmes, orientation réseaux et télécommunications}}
        \end{center}
        \vspace{5cm}

        \date{}
}

\newcommand{\troisiemepage}{
     \thispagestyle{page_de_garde}
        \vspace*{2cm}
        \begin{center}    
            {\large \textbf{HAUTE ECOLE DE LA COMMUNAUTE FRANCAISE EN HAINAUT}}\\
            {\large Département Sciences et Technologies}\\
            {\large 8A Avenue Victor Maistriau -- 7000 Mons}
        \end{center}
        \vspace{3cm}
        \begin{center}
            \fboxrule=0.3mm
            \fbox{
                \parbox{\textwidth}{
                    \vspace{1cm}
                    \centering
                    \textbf{Lorem ipsum dolor sit amet, consectetur adipiscing elit. Vivamus lacinia odio vitae}
                    \vspace{1cm}
                }
            }
        \end{center}
        \vspace{1.5cm}
        \begin{center}
            \normalsize{\textit{Travail de fin d'études réalisé en vue de l'obtention du titre de bachelier en Informatique et systèmes, orientation réseaux et télécommunications}}
        \end{center}
        \vspace{5cm}

        \date{}
}

\begin{document}
\raggedright

\premierepage

\newpage
\thispagestyle{empty} 
\mbox{}


\newpage
\pagenumbering{arabic} 
\troisiemepage

\newpage
\section*{}
\thispagestyle{empty} % Supprime le numéro de page pour cette section

\vfill % Pousse le contenu vers le bas

\hfill % Aligne le contenu à droite
\begin{minipage}{0.5\textwidth} % Définit une largeur pour le texte
    {\itshape % Utilisation de \itshape pour appliquer l'italique à tout le contenu
        Je tiens à exprimer ma profonde gratitude à toutes les personnes qui ont contribué, de près ou de loin, à la réalisation de ce projet.

        Je remercie tout particulièrement :
        

        \vspace{1cm}

}
\end{minipage}

\vspace{1cm} % Ajoute un espace en bas si nécessaire

\newpage
\tableofcontents
\newpage
\listoffigures
\newpage
\listoftables
\newpage

    Voici les différents paramètres pour mettre en forme le texte : 

    Exemple: \textbf{Texte en gras}.
    \begin{verbatim}
    \textbf{texte} : Pour mettre le texte en gras.
    \end{verbatim}
    
    Exemple: \textit{Texte en italique}.
    \begin{verbatim}
    \textit{texte} : Pour mettre le texte en italique.
    \end{verbatim}
    
    Exemple: \underline{Texte souligné}.
    \begin{verbatim}
    \underline{texte} : Pour souligner le texte.
    \end{verbatim}
    
    Exemple: \texttt{Texte en code}.
    \begin{verbatim}
    \texttt{texte} : Pour représenter du code.
    \end{verbatim}
    
    Exemple: \textsc{Texte en petites majuscules}.
    \begin{verbatim}
    \textsc{texte} : Pour mettre le texte en petites majuscules.
    \end{verbatim}
    
    Exemple: \textsf{Texte sans empattement}.
    \begin{verbatim}
    \textsf{texte} : Pour un style sans empattement.
    \end{verbatim}
    
    Exemple: \textsl{Texte en italique}.
    \begin{verbatim}
    \textsl{texte} : Pour mettre le texte en italique (autre variante).
    \end{verbatim}
    
    Exemple: \textbf{\textit{Texte en gras et italique}}.
    \begin{verbatim}
    \textbf{\textit{texte}} : Pour mettre le texte en gras et italique.
    \end{verbatim}

    Exemple: \textbf{\underline{Texte en gras et souligné}}.
    \begin{verbatim}
    \textbf{\underline{texte}} : Pour mettre le texte en gras et souligné.
    \end{verbatim}

    Exemple: \textit{\underline{Texte en italique et souligné}}.
    \begin{verbatim}
    \textit{\underline{texte}} : Pour mettre le texte en italique et souligné.
    \end{verbatim}

    Exemple: $x^6$ ou $H_2O$ RTBF$^{87}$.
    \begin{verbatim}
    Texte en exposant : Pour puissances ou indices, ex: $x^6$ ou $H_2O$ RTBF$^{87}$.
    \end{verbatim}

    Exemple: \hyperlink{RTBF}{RTBF}$^{\hyperlink{RTBF}{87}}$.
    \begin{verbatim}
    Texte avec lien : \hyperlink{RTBF}{RTBF}$^{\hyperlink{RTBF}{87}}$.
    \end{verbatim}
    
    Exemple: Référence\textsuperscript{\cite{ref1}}.
    \begin{verbatim}
    Référence\textsuperscript{\cite{ref1}} : Ajouter des références.
    \end{verbatim}

    \begin{verbatim}
        \raggedright : Pour eviter de couper les mots en fin de ligne
    \end{verbatim}
    



\newpage
\input{sections/Introduction.tex} % ajoute l'introduction
\newpage

\section{chap1}
Lorem ipsum dolor sit amet, consectetur adipiscing elit. Vivamus lacinia odio vitae vestibulum vestibulum. % ajoute le chapitre 1
\newpage
\section{Abréviations}
\underline  {Liste des abréviations :} 

\hypertarget{RTBF}{87 RTBF} : Radio-télévision belge de la communauté française. % ajoute la liste des abréviations
\newpage
\section*{Bibliographie}

\nocite{*}

\subsection*{Livres}
\printbibliography[type=book,heading=none]

\subsection*{Articles}
\printbibliography[type=article,heading=none]


 % ajoute les bibliographies

\end{document}