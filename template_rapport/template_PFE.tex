\documentclass[a4paper, 12pt]{article}
\usepackage[utf8]{inputenc}
\usepackage[T1]{fontenc}
\usepackage[french]{babel}
\usepackage{graphicx}
\usepackage{amsmath}
\usepackage{hyperref}
\usepackage{geometry}

\geometry{left=2.5cm, right=2.5cm, top=2.5cm, bottom=2.5cm}

\title{
    \begin{center}
        \includegraphics[width=0.5\textwidth]{img/logo_hehbe_tech.png}
    \end{center}
    \vspace{2cm}
    \begin{center}
        \underline{Rapport de projet de fin d'études}
    \end{center}
    \vspace{2cm}
    \begin{center}
        \fboxrule=0.3mm
        \fbox{
            \parbox{\textwidth}{
                \vspace{1cm}
                \centering
                \textbf{Titre du PFE Lorem ipsum dolor sit amet, consectetur adipiscing elit. Vivamus lacinia odio vitae}
                \vspace{1cm}
            }
        }
    \end{center}
    \vspace{2cm}
    \begin{center}
        \small Entreprise d'accueil : Nom de l'entreprise\\
    \end{center}
    \begin{center}
        \small Année académique : 2023-2024
    \end{center}
    \author{Réalisé par : author}
    \date{\today}

}

\begin{document}
\maketitle
\newpage
\begin{abstract}
    Voici un exemple de texte avec différents paramètres appliqués :
    
    \textbf{Texte en gras} : Utilisé pour mettre en évidence des mots ou des phrases importantes.
    
    \textit{Texte en italique} : Utilisé pour souligner des termes techniques ou des titres d'œuvres.
    
    \underline{Texte souligné} : Utilisé pour attirer l'attention sur des éléments spécifiques.
    
    \texttt{Texte en typewriter} : Utilisé pour représenter du code ou des commandes.
    
    \textsc{Texte en petites majuscules} : Utilisé pour des acronymes ou des noms propres.
    
    \textsf{Texte en sans-serif} : Utilisé pour un style de police sans empattement.
    
    \textsl{Texte en italique incliné} : Une autre variante de l'italique.
    
    \textbf{\textit{Texte en gras et italique}} : Combinaison de gras et italique pour une emphase supplémentaire.
    
    \textbf{\underline{Texte en gras et souligné}} : Combinaison de gras et souligné pour une emphase supplémentaire.
    
    \textit{\underline{Texte en italique et souligné}} : Combinaison d'italique et souligné pour une emphase supplémentaire.
    
    Texte en exposant : Utilisé pour les puissances ou les indices, par exemple, $x^6$ ou $H_2O$ RTBF$^{87}$
    
    Texte avec lien vers abréviations : Utilisé pour créer des liens hypertextes vers les abréviations, par exemple, \hyperlink{RTBF}{RTBF}$^{\hyperlink{RTBF}{87}}$

    
\end{abstract}

\newpage
\tableofcontents

\newpage
\section{Introduction}
Introduction de votre rapport.

\subsection{Contexte}
Lorem ipsum dolor sit amet, consectetur adipiscing elit. Vivamus lacinia odio vitae vestibulum vestibulum.


\subsection{Objectifs}
Lorem ipsum dolor sit amet, consectetur adipiscing elit. Vivamus lacinia odio vitae vestibulum vestibulum.

\newpage
\section{État de l'art}
Description de l'état de l'art.

\subsection{Revue de littérature}
Lorem ipsum dolor sit amet, consectetur adipiscing elit. Vivamus lacinia odio vitae vestibulum vestibulum.

\subsection{Travaux existants}
Lorem ipsum dolor sit amet, consectetur adipiscing elit. Vivamus lacinia odio vitae vestibulum vestibulum.

\newpage
\section{Méthodologie}
Description de la méthodologie utilisée.

\subsection{Approche}
Lorem ipsum dolor sit amet, consectetur adipiscing elit. Vivamus lacinia odio vitae vestibulum vestibulum.

\subsection{Techniques}
Lorem ipsum dolor sit amet, consectetur adipiscing elit. Vivamus lacinia odio vitae vestibulum vestibulum.

\newpage
\section{Résultats}
Présentation des résultats obtenus.

\subsection{Données}
Lorem ipsum dolor sit amet, consectetur adipiscing elit. Vivamus lacinia odio vitae vestibulum vestibulum.

\subsection{Analyse}
Lorem ipsum dolor sit amet, consectetur adipiscing elit. Vivamus lacinia odio vitae vestibulum vestibulum.

\newpage
\section{Discussion}
Discussion des résultats et interprétations.

\subsection{Interprétation}
Lorem ipsum dolor sit amet, consectetur adipiscing elit. Vivamus lacinia odio vitae vestibulum vestibulum.

\subsection{Limites}
Lorem ipsum dolor sit amet, consectetur adipiscing elit. Vivamus lacinia odio vitae vestibulum vestibulum.

\newpage
\section{Conclusion}
Conclusion et perspectives.

\subsection{Résumé}
Lorem ipsum dolor sit amet, consectetur adipiscing elit. Vivamus lacinia odio vitae vestibulum vestibulum.

\subsection{Perspectives}
Lorem ipsum dolor sit amet, consectetur adipiscing elit. Vivamus lacinia odio vitae vestibulum vestibulum.

\newpage
\section{Abréviations}
\underline  {Liste des abréviations :} 

\hypertarget{RTBF}{RTBF} : Radio-télévision belge de la communauté française.

\newpage
\bibliographystyle{plain}
\section{References}
Liste des références.



\end{document}