\newpage
\chapter{Commandes}


Voici les différents paramètres pour mettre en forme le texte : 

Exemple: \textbf{Texte en gras}.
\begin{verbatim}
\textbf{texte} : Pour mettre le texte en gras.
\end{verbatim}

Exemple: \textit{Texte en italique}.
\begin{verbatim}
\textit{texte} : Pour mettre le texte en italique.
\end{verbatim}

Exemple: \underline{Texte souligné}.
\begin{verbatim}
\underline{texte} : Pour souligner le texte.
\end{verbatim}

Exemple: \texttt{Texte en code}.
\begin{verbatim}
\texttt{texte} : Pour représenter du code.
\end{verbatim}

Exemple: \textsc{Texte en petites majuscules}.
\begin{verbatim}
\textsc{texte} : Pour mettre le texte en petites majuscules.
\end{verbatim}

Exemple: \textsf{Texte sans empattement}.
\begin{verbatim}
\textsf{texte} : Pour un style sans empattement.
\end{verbatim}

Exemple: \textsl{Texte en italique}.
\begin{verbatim}
\textsl{texte} : Pour mettre le texte en italique (autre variante).
\end{verbatim}

Exemple: \textbf{\textit{Texte en gras et italique}}.
\begin{verbatim}
\textbf{\textit{texte}} : Pour mettre le texte en gras et italique.
\end{verbatim}

Exemple: \textbf{\underline{Texte en gras et souligné}}.
\begin{verbatim}
\textbf{\underline{texte}} : Pour mettre le texte en gras et souligné.
\end{verbatim}

Exemple: \textit{\underline{Texte en italique et souligné}}.
\begin{verbatim}
\textit{\underline{texte}} : Mettre le texte en italique et souligné.
\end{verbatim}

Exemple: $x^6$ ou $H_2O$ RTBF$^{87}$.
\begin{verbatim}
Texte en exposant : Puissances/indices, ex: $x^6$ ou $H_2O$ RTBF$^{87}$.
\end{verbatim}

Exemple: \hyperlink{RTBF}{RTBF}$^{\hyperlink{RTBF}{87}}$.
\begin{verbatim}
Texte avec lien : \hyperlink{RTBF}{RTBF}$^{\hyperlink{RTBF}{87}}$.
\end{verbatim}

Exemple: Référence\textsuperscript{\cite{ref1}}.
\begin{verbatim}
Référence\textsuperscript{\cite{ref1}} : Ajouter des références.
\end{verbatim}

\begin{verbatim}
    \raggedright : Pour eviter de couper les mots en fin de ligne
\end{verbatim}


\chapter{Introduction}
Introduction de votre rapport.\footnote{Cette présente le contexte et les objectifs du rapport.}

\section{Contexte}
Lorem ipsum dolor sit amet, consectetur adipiscing elit. Vivamus lacinia odio vitae vestibulum vestibulum.

% Ajout d'une figure
\begin{figure}[!ht]
    \centering
    \includegraphics[width=8cm]{img/logo_hehbe_tech.png}
    \caption{Logo de la HEHBE}
    \label{fig:example_figure}
\end{figure}

\section{Objectifs}
Lorem ipsum dolor sit amet, consectetur adipiscing elit. Vivamus lacinia odio vitae vestibulum vestibulum.
% Ajout d'une table
\begin{table}[!ht]
    \centering
    \begin{tabular}{|c|c|c|}
        \hline
        Colonne 1 & Colonne 2 & Colonne 3 \\
        \hline
        Valeur 1 & Valeur 2 & Valeur 3 \\
        \hline
        Valeur 4 & Valeur 5 & Valeur 6 \\
        \hline
    \end{tabular}
    \caption{Exemple de tableau dans l'introduction}
    \label{tab:example_table}
    \vspace{0.2cm} % Espace entre la légende et la source
    Source :  Machin U., Exemple de tableau, Edition imaginaire, 2008.
\end{table}


