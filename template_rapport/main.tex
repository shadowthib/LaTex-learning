\documentclass[a4paper, 12pt]{article}
\usepackage[utf8]{inputenc}
\usepackage[T1]{fontenc}
\usepackage[french]{babel}
\usepackage{graphicx}
\usepackage{amsmath}
\usepackage{hyperref}
\usepackage{geometry}
\usepackage{comment}

\geometry{left=2.5cm, right=2.5cm, top=2.5cm, bottom=2.5cm}

\title{
    \begin{center}
        \includegraphics[width=0.5\textwidth]{img/logo_hehbe_tech.png}
    \end{center}
    \vspace{2cm}
    \begin{center}
        \underline{Rapport de projet de fin d'études}
    \end{center}
    \vspace{2cm}
    \begin{center}
        \fboxrule=0.3mm
        \fbox{
            \parbox{\textwidth}{
                \vspace{1cm}
                \centering
                \textbf{Lorem ipsum dolor sit amet, consectetur adipiscing elit. Vivamus lacinia odio vitae}
                \vspace{1cm}
            }
        }
    \end{center}
    \vspace{2cm}
    \begin{center}
        \small Entreprise d'accueil : Lorem ipsum dolor\\
    \end{center}
    \begin{center}
        \small Année académique : lorem ipsum
    \end{center}
    \author{Réalisé par : lorem ipsum}
    \date{\today}

}

\begin{document}
\maketitle
\newpage
\begin{abstract}
    \begin{verbatim}
    \textbf{Texte en gras} : Pour mettre en évidence.
    \end{verbatim}

    \begin{verbatim}
    \textit{Texte en italique} : Pour souligner des termes.
    \end{verbatim}

    \begin{verbatim}
    \underline{Texte souligné} : Pour attirer l'attention.
    \end{verbatim}

    \begin{verbatim}
    \texttt{Texte en typewriter} : Pour représenter du code.
    \end{verbatim}

    \begin{verbatim}
    \textsc{Texte en petites majuscules} : Pour acronymes ou noms propres.
    \end{verbatim}

    \begin{verbatim}
    \textsf{Texte en sans-serif} : Pour un style sans empattement.
    \end{verbatim}

    \begin{verbatim}
    \textsl{Texte en italique incliné} : Variante de l'italique.
    \end{verbatim}

    \begin{verbatim}
    \textbf{\textit{Texte en gras et italique}} : Emphase supplémentaire.
    \end{verbatim}

    \begin{verbatim}
    \textbf{\underline{Texte en gras et souligné}} : Emphase supplémentaire.
    \end{verbatim}

    \begin{verbatim}
    \textit{\underline{Texte en italique et souligné}} : Emphase supplémentaire.
    \end{verbatim}

    \begin{verbatim}
    Texte en exposant : Pour puissances ou indices, ex: $x^6$ ou $H_2O$ RTBF$^{87}$.
    \end{verbatim}

    \begin{verbatim}
    Texte avec lien : \hyperlink{RTBF}{RTBF}$^{\hyperlink{RTBF}{87}}$.
                      Exemple dans les références
    \end{verbatim}

    \begin{verbatim}
    Référence\textsuperscript{\cite{ref1}} : Ajouter des références.
                                             Exemple dans la bibliographie.
    \end{verbatim}

    \begin{verbatim}
    \begin{comment} au debut du bloc et \end{comment} a la fin pour commenter.
    \end{verbatim}

    
    
\end{abstract}

\newpage
\tableofcontents

\newpage
\input{sections/Introduction.tex} % ajoute l'introduction
\newpage

\section{chap1}
Lorem ipsum dolor sit amet, consectetur adipiscing elit. Vivamus lacinia odio vitae vestibulum vestibulum. % ajoute le chapitre 1
\newpage
\section{Abréviations}
\underline  {Liste des abréviations :} 

\hypertarget{RTBF}{87 RTBF} : Radio-télévision belge de la communauté française. % ajoute la liste des abréviations
\newpage
\section*{Bibliographie}

\nocite{*}

\subsection*{Livres}
\printbibliography[type=book,heading=none]

\subsection*{Articles}
\printbibliography[type=article,heading=none]


 % ajoute les bibliographies




\end{document}